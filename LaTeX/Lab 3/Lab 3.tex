%%%%%%%%%%%%%%%%%%%%%%%%%%%%%%%%%%%%%%%%%
%
% CMPT xxx
% Spring 2019
% Lab One
%
%%%%%%%%%%%%%%%%%%%%%%%%%%%%%%%%%%%%%%%%%

%%%%%%%%%%%%%%%%%%%%%%%%%%%%%%%%%%%%%%%%%
% Short Sectioned Assignment
% LaTeX Template
% Version 1.0 (5/5/12)
%
% This template has been downloaded from: http://www.LaTeXTemplates.com
% Original author: % Frits Wenneker (http://www.howtotex.com)
% License: CC BY-NC-SA 3.0 (http://creativecommons.org/licenses/by-nc-sa/3.0/)
% Modified by Alan G. Labouseur  - alan@labouseur.com
% Modified further by Robert E. Liskin - robert.liskin1@marist.edu
%%%%%%%%%%%%%%%%%%%%%%%%%%%%%%%%%%%%%%%%%

%----------------------------------------------------------------------------------------
%	PACKAGES AND OTHER DOCUMENT CONFIGURATIONS
%----------------------------------------------------------------------------------------

\documentclass[letterpaper, 10pt,DIV=13]{scrartcl} 

\usepackage[T1]{fontenc} % Use 8-bit encoding that has 256 glyphs
\usepackage[english]{babel} % English language/hyphenation
\usepackage{amsmath,amsfonts,amsthm,xfrac} % Math packages
\usepackage{sectsty} % Allows customizing section commands
\usepackage{graphicx}
\usepackage[lined,linesnumbered,commentsnumbered]{algorithm2e}
\usepackage{listings}
\usepackage{parskip}
\usepackage{lastpage}

\allsectionsfont{\normalfont\scshape} % Make all section titles in default font and small caps.

\usepackage{fancyhdr} % Custom headers and footers
\pagestyle{fancyplain} % Makes all pages in the document conform to the custom headers and footers

\fancyhead{} % No page header - if you want one, create it in the same way as the footers below
\fancyfoot[L]{} % Empty left footer
\fancyfoot[C]{} % Empty center footer
\fancyfoot[R]{page \thepage\ of \pageref{LastPage}} % Page numbering for right footer

\renewcommand{\headrulewidth}{0pt} % Remove header underlines
\renewcommand{\footrulewidth}{0pt} % Remove footer underlines
\setlength{\headheight}{13.6pt} % Customize the height of the header

\numberwithin{equation}{section} % Number equations within sections (i.e. 1.1, 1.2, 2.1, 2.2 instead of 1, 2, 3, 4)
\numberwithin{figure}{section} % Number figures within sections (i.e. 1.1, 1.2, 2.1, 2.2 instead of 1, 2, 3, 4)
\numberwithin{table}{section} % Number tables within sections (i.e. 1.1, 1.2, 2.1, 2.2 instead of 1, 2, 3, 4)

\setlength\parindent{0pt} % Removes all indentation from paragraphs.

\binoppenalty=3000
\relpenalty=3000

%----------------------------------------------------------------------------------------
%	TITLE SECTION
%----------------------------------------------------------------------------------------

\newcommand{\horrule}[1]{\rule{\linewidth}{#1}} % Create horizontal rule command with 1 argument of height

\title{	
   \normalfont \normalsize 
   \textsc{CMPT 424 - Fall 2019 - Dr. Labouseur} \\[10pt] % Header stuff.
   \horrule{0.5pt} \\[0.25cm] 	% Top horizontal rule
   \huge Lab Three  \\     	    % Assignment title
   \horrule{0.5pt} \\[0.25cm] 	% Bottom horizontal rule
}

\author{Robert Liskin \\ \normalsize Robert.Liskin1@Marist.edu}

\date{\normalsize\today} 	% Today's date.

\begin{document}
\maketitle % Print the title

%----------------------------------------------------------------------------------------
%   start PROBLEM ONE
%----------------------------------------------------------------------------------------
\section{Question One}

\subsection{Explain the difference between internal and external fragmentation.}

Internal and external fragmentation are mutually-exclusive consequences of how memory is allocated. Internal fragmentation occurs when statically sized memory is assigned to a program. If the program does not use all of the memory it has been allocated, that leftover space is wasted, or 'internally fragmented.' Because it is bound by the size of the allocation, nothing can be done with it. External fragmentation, on the other hand, occurs in variably sized memory allocation.
\par
External fragmentation occurs when, as programs are assigned just the memory they need and subsequently finish, empty holes start to appear. A program might need X amount of memory to run, and adding the empty holes together might sum to Y memory where Y > X, but because it is not contiguous, that program cannot run. To combat this issue, many operating systems have programs that will defragment memory, usually a hard drive. While we're concerned more with 'RAM,' the same principles still apply.


%----------------------------------------------------------------------------------------
%   end PROBLEM ONE
%----------------------------------------------------------------------------------------
\pagebreak
%----------------------------------------------------------------------------------------
%   start PROBLEM TWO
%----------------------------------------------------------------------------------------
\section{Question Two}

\subsection{Given five (5) memory partitions of 100KB, 500KB, 200KB, 300KB, and 600KB (in that order), how would optimal, first-fit, best-fit, and worst-fit algorithms place processes of 212KB, 417KB, 112KB, and 426KB (in that order)?}
\par
Assuming none of these processes complete while others are being assigned memory...
\par
First-Fit: The 212KB would be assigned the 500KB, the 417KB would be assigned the 600KB, the 112KB would be assigned the 200KB, and the 426KB would have to wait. It happens in this order because First-Fit grants the largest hole that is available, so a memory hole might be big enough, but if it is unavailable, as what happens to the 426KB, then it has to wait.
\par
Best-Fit: The 212KB would be assigned the 300KB, the 417KB would be assigned the 500KB, the 112KB would be assigned the 200KB, and the 426KB would be assigned the 600KB. It happens in this order because Best-Fit actually attempts to optimize memory allocation by proactively reducing internal fragmentation. In this instance, Best-Fit is more optimal than First-Fit.
\par
Worst-Fit: The 212KB would be assigned the 600KB, the 417KB would be assigned the 500KB, the 112KB would be assigned the 300KB, and the 426KB would again, have to wait. It happens in the order because Worst-Fit assigns the largest first available hole, disregarding the actual program size.


%----------------------------------------------------------------------------------------
%   end PROBLEM Two
%----------------------------------------------------------------------------------------

\pagebreak

\end{document}
